\documentclass{article}
\usepackage{amssymb}
\usepackage{graphicx}
\usepackage{hyperref}
\usepackage[left=1.5cm, right=1.5cm]{geometry}

\hypersetup{
    colorlinks=true,
    linkcolor=blue,
    filecolor=magenta,      
    urlcolor=cyan,
}

\makeatletter
\renewcommand\subsection{\@startsection{subsection}{2}{\z@}%
                                     {-3.25ex\@plus-1ex \@minus-.2ex}%
                                     {1.5ex \@plus.2ex}%
                                     {\normalfont\normalsize\bfseries}}
\makeatother

\renewcommand{\contentsname}{Indice} % Change the title of the table of contents

% chktex-file 44
\title{TuTubi}
\author{}
\date{}

\renewcommand{\labelenumi}{\arabic{enumi}.}
\renewcommand{\labelenumii}{\arabic{enumi}.\arabic{enumii}.}
\renewcommand{\labelenumiii}{\arabic{enumi}.\arabic{enumii}.\arabic{enumiii}.}
\renewcommand{\labelenumiv}{\arabic{enumi}.\arabic{enumii}.\arabic{enumiii}.\arabic{enumiv}.}
\begin{document}

\maketitle

\tableofcontents

\newpage
\section{\label{sec:Requisiti}Requisiti}
\begin{enumerate}
    \item\label{sec:RequisitiUtente} Utente
    \begin{enumerate}
        \item nome: \hyperref[sec:TipoStringS]{StringS}
        \item dataIscrizione: DataOra
        \item I \hyperref[sec:RequisitiVideo]{video} pubblicati dall'utente
        \item Le \hyperref[sec:RequisitiValutazione]{valutazioni} che ha espresso
        \item I \hyperref[sec:RequisitiCommento]{commenti} che ha pubblicato
    \end{enumerate}
    \item\label{sec:RequisitiVideo} Video
    \begin{enumerate}
        \item titolo: \hyperref[sec:TipoStringS]{StringS}
        \item La durata (operazione)
        \item descrizione: \hyperref[sec:TipoStringL]{StringL}
        \item nomeFile: \hyperref[sec:TipoStringM]{StringM}
        \item La \hyperref[sec:RequisitiCategoria]{categoria} (unica)
        \item I \hyperref[sec:RequisitiTag]{tag} (almeno uno)
        \item Se è un video in risposta a un altro video:
        \begin{enumerate}
            \item Il video a cui risponde
            \item Non può essere in risposta a un video pubblicato dallo stesso utente (vincolo)
        \end{enumerate}
        \item Numero di visualizzazioni (operazione)
        \item Media valutazioni (operazione)
        \item Se è stato \hyperref[sec:RequisitiVideoCensurato]{censurato}
        \item Numero di valutazioni (operazione)
    \end{enumerate}
    \item\label{sec:RequisitiCategoria} Categoria
    \begin{enumerate}
        \item nome: \hyperref[sec:TipoStringS]{StringS}
    \end{enumerate}
    \item\label{sec:RequisitiTag} Tag
    \begin{enumerate}
        \item nome: \hyperref[sec:TipoStringS]{StringS}
    \end{enumerate}
    \item\label{sec:RequisitiCronologia} Cronologia
    \begin{enumerate}
        \item L'\hyperref[sec:RequisitiUtente]{utente} che ha effettuato la visualizzazione
        \item Il \hyperref[sec:RequisitiVideo]{video} visualizzato
        \item dataVisualizzazione: DataOra
    \end{enumerate}
    \item\label{sec:RequisitiValutazione} Valutazione
    \begin{enumerate}
        \item voto: \hyperref[sec:TipoVoto]{Voto}
        \item L'utente che ha pubblicato il video non può votarlo (vincolo)
        \item Gli altri utenti possono votare un video al più una sola volta (vincolo)
        \item Un utente può esprimere una valutazione solo su un video che ha visualizzato (vincolo)
        \item L'\hyperref[sec:RequisitiUtente]{utente} che ha espresso la valutazione
        \item Il \hyperref[sec:RequisitiVideo]{video} valutato
    \end{enumerate}
    \item\label{sec:RequisitiCommento} Commento
    \begin{enumerate}
        \item testo: \hyperref[sec:TipoStringL]{StringL}
        \item dataPubblicazione: DataOra
        \item Gli utenti possono esprimere più commenti sullo stesso video (vincolo)
        \item Un utente può scrivere un commento solo su un video che ha visualizzato (vincolo)
    \end{enumerate}
    \item\label{sec:RequisitiPlaylist} Playlist
    \begin{enumerate}
        \item I \hyperref[sec:RequisitiVideo]{video} ordinati
        \begin{enumerate}
            \item DataOra di inserimento in playlist
        \end{enumerate}
        \item nome: \hyperref[sec:TipoStringS]{StringS}
        \item dataCreazione: DataOra
        \item stato: \hyperref[sec:TipoStatoPlaylist]{StatoPlaylist}
    \end{enumerate}
    \item\label{sec:RequisitiVideoCensurato} Video Censurato
    \begin{enumerate}
        \item La \hyperref[sec:RequisitiMotivazioneCensura]{motivazione} della censura
        \item Non può essere visualizzato (vincolo)
        \item Non può essere valutato (vincolo)
        \item Non può essere commentato (vincolo)
        \item Non può essere inserito in una \hyperref[sec:RequisitiPlaylist]{playlist} (vincolo)
        \item Non può essere restituito come risultato di una ricerca (vincolo)
        \item Una volta censurato non può più essere ripristinato (vincolo)
        \item commentoAggiuntivo: \hyperref[sec:TipoStringL]{StringL}
    \end{enumerate}
    \item\label{sec:RequisitiMotivazioneCensura} MotivazioneCensura
    \begin{enumerate}
        \item nome: \hyperref[sec:TipoStringS]{StringS}
        \item descrizione: \hyperref[sec:TipoStringL]{StringL}
    \end{enumerate}
\end{enumerate}

\newpage

\section{\label{sec:TipoDiDato}Specifica dei tipi di dato}
\begin{enumerate}
    \item\label{sec:TipoInteroGZ} InteroGZ$: $Intero $>$ 0
    \item\label{sec:TipoInteroGEZ} InteroGEZ$: $Intero $\geq$ 0
    \item\label{sec:TipoStringS} StringS$: $Stringa di 75 caratteri
    \item\label{sec:TipoStringM} StringM$: $Stringa di 600 caratteri
    \item\label{sec:TipoStringL} StringL$: $Stringa di 4200 caratteri
    \item\label{sec:TipoVoto} Voto$: $Intero tra 0 e 5
    \item\label{sec:TipoRealeGEZ} RealeGEZ$: $Reale $\geq$ 0
    \item\label{sec:TipoDurata} Durata$: $Durata in formato hh:mm:ss
    \item\label{sec:TipoStatoPlaylist} StatoPlaylist$: $\{Pubblica, Privata\}
\end{enumerate}

\newpage

\section{\label{sec:Operazioni}Specifica delle operazioni}

\begin{enumerate}
    \item\label{sec:OperazioniUtente} L'Utente deve poter:
    \begin{enumerate}
        \item pubblicare video
        \item visualizzare video
        \item esprimere valutazioni sui video
        \item esprimere commenti testuali sui video
        \item pubblicare playlist
        \item ottenere le playlist pubbliche di un utente
        \item iscriversi
        \item data una categoria, un insieme di tag, la valutazione v tra 0 e 5, deve restituire tutti i video disponibili per quella categoria, con almeno uno dei tag specificati e una valutazione media >= v (se non ha valutazioni deve essere restituito lo stesso)
        \item la ricerca data una categoria che hanno il numero maggiore di video in risposta
    \end{enumerate}
    \item\label{sec:OperazioniRedazione} La Redazione deve poter:
    \begin{enumerate}
        \item censurare video
    \end{enumerate}
\end{enumerate}

\newpage

\section{\label{sec:DiagrammaDelleClassi}Diagramma delle classi}
\begin{figure}[h]
    \centering
    \includegraphics[width=1\textwidth]{../Diagrammi/diagramma delle classi.pdf}
    \caption{Diagramma delle classi}
\end{figure}

\newpage

\end{document}